\documentclass[a4paper]{article}
\usepackage{ctex}
\usepackage{multicol}
\usepackage{fancyhdr}
\usepackage{color}
\usepackage{CJK}
\usepackage{amsmath}
\usepackage{graphicx}
\usepackage{algorithm}%写算法时需要调用的包
\usepackage{algorithmic}%写算法时需要调用的包
\usepackage{setspace}%修改行间距需要调用的包
\definecolor{gray}{RGB}{192,192,192}%灰色设置
\title{\heiti \Large 我的五子棋AI果然有问题}%题目
\author{\songti \small 韩梓辰\ 夏星晨\ 周贤玮\ 赵云龙\ 张坤龙}%作者信息
\pagestyle{fancy}
\lhead{\textcolor{gray} {Computer Science 计算机科学}}
\rhead{\textcolor{gray} {\thepage}}
\renewcommand{\headrulewidth}{0.4pt}
\begin{document}
\maketitle %标题
\thispagestyle{fancy} %页眉
\lhead{\textcolor{gray} {计算机科学}}
\rhead{\textcolor{gray} {DOI:xxxxxx}}
\renewcommand{\headrulewidth}{0.4pt}
\begin{abstract}
    近年来,AlphaGo在棋坛上打遍天下无敌手,甚至进军电子竞技行业,人工智能在发展到今天,人类在竞技体育领域可能越来越不是他们的对手。但是,显然光对胜利的渴求并不新颖,因为人工智能现在越来越多的在各个领域聪明,从以前的人工智障变成了人工智能,在去年,日本一个公司开发了一款人工智能,号称史上最弱人工智能,这个人工智能在几百万次的游戏对战中只获取了1000次的胜利,无论人类如何放水,这个人工智能反倒越来越弱。于是放弃原有的老套人工智能思路,改为设计“人工智障”成为了一个全新的设计思路。\par
该五子棋“人工智障”将基于Python编程语言,通过数学建模,博弈树,神经网络等算法实现。使用pytorch工具,CUDA加速实现矩阵运算的优化,更加优秀的卷积神经网络设计等方法对其进行进一步的优化。最后,在通过大量的人机对战、机机对战、预设对战的数据的学习下,该人工智障已具备一定的计算机科学技术上的智能水平,具有了一定的研究与使用意义。
    \par\textbf{关键词:\ }人工智能,五子棋,神经网络,人工智障,TensorFlow
\end{abstract}
\setlength{\baselineskip}{20pt}
\tableofcontents  %表示目录部分开始
\newpage
    \begin{multicols}{2}
    \section{引言}
    近年来,人工智能的火热程度越来越高,我们几乎在各行各业都可以遇到人工智能,同时我们也可以利用人工智能帮助我们干很多事情。随着人工智能的发展,我们也发现,人工智能在很多领域超过了人类本身,2016年3月,谷歌研发的人工智能--阿尔法狗与围棋世界冠军、职业九段棋手李世石进行围棋人机大战,以4比1的总比分获胜,震惊了棋坛;2016年末2017年初,该程序在中国棋类网站上以“大师”(Master)为注册账号与中日韩数十位围棋高手进行快棋对决,连续60局无一败绩,当人们知晓的时候,无不对人工智能的力量感到佩服;2017年5月,在中国乌镇围棋峰会上,它与排名世界第一的世界围棋冠军柯洁对战,以3比0的总比分获胜,取得了围棋界的王冠。围棋界公认阿尔法围棋的棋力已经超过人类职业围棋顶尖水平。由于在战胜人类方面,人工智能越来越强,我们对这个方面觉得研究意义并不会特别大了,所以我们决定转换方向,即通过反向思路实现,将人工智能彻底做成另一个新的方向,即人工智障。我们计划设计一款可以不断的被人类战胜的机器,无论人类如何放水都可以输掉整个比赛。\par
    整体思路来源于日本的黑白棋人工智能项目\cite{ref1}
    \section{相关工作}
    整体项目思路是通过是将整个棋盘视作矩阵,通过
    \section{详细的实现}

    \section{验证}

    \section{结束语}

    \newpage
    \addcontentsline{toc}{section}{参考文献}
    \begin{thebibliography}{30}%参考文献
    \bibitem{ref1}{naka. J., The weakest Othello, Takujin Yoshida.Thoroughly dig into the inside of the development!(2019-7-25) [2020-09-01]https://ai-trend.jp/business-article/interview/othello-cto-interview}
    \bibitem{ref2}{李金洪\ 深度学习之TensorFlow\ [M]\ .北京.机械工业出版社, 2018-3}
    \end{thebibliography}
    \newpage
\end{multicols}
\end{document}
